\documentclass[UTF8, a4paper, 12pt]{ctexart}

% =========================================================
% 宏包引入
% =========================================================
\usepackage{geometry}
\geometry{left=2.5cm,right=2.5cm,top=2.5cm,bottom=2.5cm}
\usepackage{amsmath,amssymb,amsfonts,bm}
\usepackage{mathrsfs}
\usepackage{graphicx}
\usepackage{subcaption}
\usepackage{booktabs}
\usepackage{algorithm}
\usepackage{algorithmic}
\usepackage{xcolor}
\usepackage{hyperref}

\begin{document}

% =========================================================
% 第一份文档:笔试预测
% =========================================================
\begin{center}
    \LARGE \textbf{西安交大少年班复试笔试的核心选拔逻辑}
\end{center}
\vspace{1em}

西安交大少年班复试笔试的核心选拔逻辑,是通过现学现考模式,筛选具备超强即时学习能力、抽象逻辑思维、知识迁移与创新应用潜能的拔尖少年,而非考查学生提前学习大学知识的体量。所有预测考点均严格贴合三大核心原则:一是定义自治,可通过短篇幅书面材料讲清核心规则,适配现学现考形式;二是与中学知识有明确边界,最大程度规避“提前刷题补课”的优势,保证选拔公平性;三是具备层级化出题空间,可从例题定义应用进阶到综合推导、创新建模,能精准区分学生思维层次。


\section*{一、数理思维(数学)考点预测}

\subsection*{(一)核心高频考点(大)}

\subsubsection*{1. 数学分析入门:极限与连续、导数与微分的公理化推导}
\textbf{详细内容(定义、公式与含义):}

\noindent \textbf{1)数列极限($\varepsilon$-$N$ 定义):}设数列 $\{a_n\}$,若存在实数 $L$,使得对任意 $\varepsilon>0$,存在正整数 $N$,当 $n\ge N$ 时恒有
\[
|a_n-L|<\varepsilon,
\]
则称 $\lim_{n\to\infty}a_n=L$。

\noindent \textbf{含义:}$\varepsilon$ 表示允许误差;$N$ 表示从第 $N$ 项开始,所有项都落在以 $L$ 为中心、半径为 $\varepsilon$ 的区间内。

\vspace{0.5em}
\noindent \textbf{2)函数极限($\varepsilon$-$\delta$ 定义):}设函数 $f(x)$ 在点 $a$ 的去心邻域有定义。若存在实数 $L$,使得对任意 $\varepsilon>0$,存在 $\delta>0$,当 $0<|x-a|<\delta$ 时恒有
\[
|f(x)-L|<\varepsilon,
\]
则称 $\lim_{x\to a}f(x)=L$。

\noindent \textbf{含义:}$\delta$ 控制 ``输入'' $x$ 距离 $a$ 的接近程度;$\varepsilon$ 控制 ``输出'' $f(x)$ 距离 $L$ 的接近程度。

\vspace{0.5em}
\noindent \textbf{3)连续(用极限定义):}若 $f(a)$ 有定义,且
\[
\lim_{x\to a}f(x)=f(a),
\]
则称 $f$ 在 $a$处连续。

\vspace{0.5em}
\noindent \textbf{4)导数(差商极限定义):}函数 $f$ 在点 $a$ 的导数定义为
\[
 f'(a)=\lim_{h\to 0}\frac{f(a+h)-f(a)}{h},
\]
若极限存在。

\noindent \textbf{含义:}差商 $\frac{f(a+h)-f(a)}{h}$ 表示 ``平均变化率'';当 $h\to 0$ 时趋于 ``瞬时变化率'',也是曲线在点 $a$ 的切线斜率。

\vspace{0.5em}
\noindent \textbf{5)微分:}若 $f$ 在 $a$ 处可导,则当 $\Delta x$ 很小时有近似分解
\[
\Delta y=f(a+\Delta x)-f(a)=f'(a)\,\Delta x+o(\Delta x),\quad (\Delta x\to 0).
\]
定义微分 $\mathrm{d}y=f'(a)\,\mathrm{d}x$,其中 $\mathrm{d}x$ 可看作 ``可自由取的小增量''。

\vspace{0.75em}
\noindent \textbf{【核心注意事项】:}
\begin{itemize}
    \item \textbf{连续与可导的关系:}可导必连续,连续不一定可导(如 $f(x)=|x|$ 在 $x=0$ 处连续但不可导)。
    \item \textbf{极限的唯一性:}若极限存在,则极限必唯一。
    \item \textbf{导数的公理化:}在“现学现考”中,常要求用 $\varepsilon$-$\delta$ 语言证明简单的极限,或通过导数定义推导公式,而非直接套用求导法则。
    \item \textbf{局部性质:}极限反映的是点邻域内的趋势,与该点处的具体取值无关(除非讨论连续性)。
\end{itemize}

\vspace{0.75em}
\noindent \textbf{例题 1(用定义证明极限):}证明 $\displaystyle \lim_{x\to 1}\frac{x^2-1}{x-1}=2$。

\noindent \textbf{解:}对任意 $\varepsilon>0$,我们要寻找 $\delta>0$,使得当 $0<|x-1|<\delta$ 时,有
\[
\left|\frac{x^2-1}{x-1}-2\right| < \varepsilon.
\]
由于 $x\ne 1$,$\frac{x^2-1}{x-1}=x+1$。
则 $|(x+1)-2|=|x-1|$。
只需取 $\delta=\varepsilon$,则当 $0<|x-1|<\delta$ 时,有 $|x-1|<\varepsilon$,即原式成立。

\vspace{0.75em}
\noindent \textbf{例题 2(导数定义与高阶思维):}设 $f(x)$ 在 $x=0$ 处可导,且 $f(0)=0$,求极限 $\displaystyle \lim_{x\to 0}\frac{f(x^2)-f(0)}{x^2+x^3}$。

\noindent \textbf{解:}由导数定义,$f'(0)=\lim_{h\to 0}\frac{f(h)-f(0)}{h}$。
原式可变形为:
\[
\lim_{x\to 0}\frac{f(x^2)-f(0)}{x^2} \cdot \frac{x^2}{x^2+x^3} = \lim_{x\to 0}\frac{f(x^2)-f(0)}{x^2} \cdot \frac{1}{1+x} = f'(0) \cdot 1 = f'(0).
\]

\vspace{0.75em}
\noindent \textbf{例题 3(连续性与参数判定):}已知 $f(x) = \begin{cases} \frac{\sin x}{x}, & x < 0 \\ ax+b, & x \ge 0 \end{cases}$。若 $f(x)$ 在 $x=0$ 处可导,求 $a,b$ 的值。

\noindent \textbf{解:}1. 必须连续:$\lim_{x\to 0^-} f(x) = \lim_{x\to 0^-} \frac{\sin x}{x} = 1$;$f(0)=b$。故 $b=1$。
2. 左导数:$f'_-(0) = \lim_{h\to 0^-} \frac{\frac{\sin h}{h}-1}{h} = \lim_{h\to 0^-} \frac{\sin h - h}{h^2} = 0$(利用泰勒展开 $\sin h \approx h - h^3/6$)。
3. 右导数:$f'_+(0) = a$。
故 $a=0, b=1$。

\vspace{0.75em}
\noindent \textbf{核心理由:}

\textbf{往年命题延续性极强:}2024、2025连续两年将极限定义作为现学现考核心大题,是大学数学与中学数学的核心思维分水岭,完美适配现学现考的选拔逻辑;

\textbf{选拔属性拉满:}中学阶段仅涉及极限的直观理解,考察再进一步的陌生的公理化逻辑体系,能真正考查学生对抽象数学语言的快速理解、严谨逻辑推导能力,而非死记硬背;

\textbf{出题层级丰富:}可从例题的“用定义证明极限/连续”,进阶到可导性判定、结合不等式的综合证明,完全适配近年“仅几道大题”的题型设置,能清晰区分不同层次学生的思维能力;

\textbf{学科适配性:}是大学所有理工科的核心数学例题,完全契合少年班后续本硕博贯通培养的核心能力要求。

\subsubsection*{2. 抽象代数入门:群论例题(群的公理化定义、基本性质、子群与逆元、同态初步)}
\textbf{详细内容(定义、公式与含义):}

\noindent \textbf{1)群的定义:}设非空集合 $G$ 上定义了二元运算 $\cdot:G\times G\to G$,若满足以下四条公理,则称 $(G,\cdot)$ 为群:
\begin{enumerate}
\item \textbf{封闭性:}对任意 $a,b\in G$,有 $a\cdot b\in G$;
\item \textbf{结合律:}对任意 $a,b,c\in G$,有 $(a\cdot b)\cdot c=a\cdot(b\cdot c)$;
\item \textbf{单位元:}存在 $e\in G$,使得对任意 $a\in G$,有 $e\cdot a=a\cdot e=a$;
\item \textbf{逆元:}对任意 $a\in G$,存在 $a^{-1}\in G$,使得 $a\cdot a^{-1}=a^{-1}\cdot a=e$。
\end{enumerate}

\noindent \textbf{含义:}群是带有“运算”的集合,该运算满足“封闭、结合、存在单位元、每个元素都有逆元”四大性质,是描述“对称性”和“变换”的基本代数结构。

\vspace{0.5em}
\noindent \textbf{2)阿贝尔群:}若群 $G$ 的运算还满足交换律,即对任意 $a,b\in G$ 有 $a\cdot b=b\cdot a$,则称 $G$ 为阿贝尔群(交换群)。

\vspace{0.5em}
\noindent \textbf{3)子群:}设 $(G,\cdot)$ 为群,$H\subseteq G$ 非空。若 $(H,\cdot)$ 本身也构成群,则称 $H$ 为 $G$ 的子群,记作 $H\le G$。

\noindent \textbf{子群判定定理:}非空子集 $H$ 是 $G$ 的子群当且仅当对任意 $a,b\in H$,有 $ab^{-1}\in H$。

\vspace{0.5em}
\noindent \textbf{4)同态与同构:}设 $(G,\cdot)$、$(H,*)$ 为群。若映射 $\varphi:G\to H$ 满足
\[
\varphi(a\cdot b)=\varphi(a)*\varphi(b),\quad\forall a,b\in G,
\]
则称 $\varphi$ 为群同态。若 $\varphi$ 还是双射,则称 $\varphi$ 为群同构。

\noindent \textbf{含义:}同态保持运算结构;同构表示两个群在代数意义上完全相同。

\vspace{0.75em}
\noindent \textbf{例题 1(判定是否构成群):}设 $S=\{\,1,-1,i,-i\,\}$,复数乘法是否使 $S$ 构成群?

\noindent \textbf{解:}列出乘法表:
\[
\begin{array}{c|cccc}
\cdot & 1 & -1 & i & -i\\\hline
1 & 1 & -1 & i & -i\\
-1 & -1 & 1 & -i & i\\
i & i & -i & -1 & 1\\
-i & -i & i & 1 & -1
\end{array}
\]
- 封闭:表中所有结果仍在 $S$ 中;
- 结合律:复数乘法本身满足结合律;
- 单位元:$1$ 满足 $1\cdot a=a$;
- 逆元:$1^{-1}=1$,$(-1)^{-1}=-1$,$i^{-1}=-i$,$(-i)^{-1}=i$。

因此 $(S,\cdot)$ 是群,且是阿贝尔群。

\vspace{0.75em}
\noindent \textbf{例题 2(子群判定):}在整数加法群 $(\mathbb Z,+)$ 中,判断 $H=2\mathbb Z=\{\,2k\mid k\in\mathbb Z\,\}$ 是否为子群。

\noindent \textbf{解:}取任意 $a=2m,b=2n\in H$,则
\[
a-b=2m-2n=2(m-n)\in H.
\]
由子群判定定理,$H$ 是 $\mathbb Z$ 的子群。

\vspace{0.75em}
\noindent \textbf{例题 3(证明群的单位元唯一):}证明群 $G$ 的单位元唯一。

\noindent \textbf{解:}设 $e,e'$ 都是 $G$ 的单位元。则
\[
e=e\cdot e'=e' \quad (\text{因为 }e'\text{ 是单位元}).
\]
故单位元唯一。

\vspace{0.75em}
\noindent \textbf{核心理由:}

\textbf{近年命题核心热点:}2025年复试直接考查群论相关的新运算、结合律、逆元证明题,是当前数理思维模块的核心区分度题型;

\textbf{绝对公平的选拔性:}群论是完全脱离中学数学知识体系的内容,几乎无初中生/普通高中生提前系统学习,能彻底规避提前补课的优势,100\%聚焦学生的抽象思维、规则理解与迁移应用能力,完全贴合“现学现考”的核心目标;

\textbf{命题灵活性极强:}可先给定群的四大公理,再从“判定集合是否构成群”的例题题,进阶到群的性质证明、子群构造、简单同态映射的综合题,层层递进的出题空间完全适配大题考查需求;

\textbf{学科价值突出:}是现代数学、物理、计算机科学的核心例题,契合少年班拔尖创新人才的培养方向。

\subsubsection*{3. 凸分析与不等式初步:琴生不等式、赫尔德不等式的定义与综合应用}
\textbf{详细内容(定义、公式与含义):}

\noindent \textbf{1)凸函数的定义:}区间 $I$ 上的函数 $f:I\to\mathbb R$ 若满足对任意 $x,y\in I$ 与任意 $\lambda\in[0,1]$,都有
\[
f(\lambda x+(1-\lambda)y)\le \lambda f(x)+(1-\lambda)f(y),
\]
则称 $f$ 在 $I$ 上为凸函数。

\noindent \textbf{几何含义:}函数图像在任意两点连线(弦)之下。

\vspace{0.5em}
\noindent \textbf{2)琴生不等式(Jensen):}若 $f$ 在区间上凸,且 $\lambda_i\ge 0,\ \sum_{i=1}^n\lambda_i=1$,则
\[
f\Bigl(\sum_{i=1}^n \lambda_i x_i\Bigr)\le \sum_{i=1}^n \lambda_i f(x_i).
\]

\noindent \textbf{含义:}凸函数作用在“加权平均”上,不超过“加权平均后的函数值”。常用来把复杂表达式“压”成更简单的平均结构。

\vspace{0.5em}
\noindent \textbf{3)赫尔德不等式(H\"older):}设 $p,q>1$ 且 $\frac1p+\frac1q=1$,则对任意实数列 $\{a_i\},\{b_i\}$ 有
\[
\sum_{i=1}^n |a_i b_i|\le \Bigl(\sum_{i=1}^n |a_i|^p\Bigr)^{1/p}\Bigl(\sum_{i=1}^n |b_i|^q\Bigr)^{1/q}.
\]

\noindent \textbf{特殊情形:}当 $p=q=2$ 时即柯西--施瓦茨不等式
\[
\Bigl(\sum_{i=1}^n a_i b_i\Bigr)^2\le \Bigl(\sum_{i=1}^n a_i^2\Bigr)\Bigl(\sum_{i=1}^n b_i^2\Bigr).
\]

\noindent \textbf{含义:}把“乘积求和”控制在两个范数的乘积之内,是处理估计与最值问题的核心工具。

\vspace{0.75em}
\noindent \textbf{例题 1(用 Jensen 证明经典不等式):}设 $a,b,c>0$,证明
\[
\frac{1}{a} + \frac{1}{b} + \frac{1}{c} \ge \frac{9}{a+b+c}.
\]

\noindent \textbf{解:}函数 $f(x)=\frac{1}{x}$ 在 $(0,\infty)$ 上二阶导数为 $f''(x)=\frac{2}{x^3}>0$,故 $f$ 凸。由 Jensen(等权 $\lambda_i=1/3$)得
\[
\frac{1}{3}\Bigl(\frac{1}{a}+\frac{1}{b}+\frac{1}{c}\Bigr)\ge \frac{1}{\frac{a+b+c}{3}}=\frac{3}{a+b+c}.
\]
两边同乘 3 即得结论。

\vspace{0.75em}
\noindent \textbf{例题 2(H\"older 的直接应用):}设 $a,b,c\ge 0$,证明
\[
(a+b+c)^2\le 3(a^2+b^2+c^2).
\]

\noindent \textbf{解:}取 $p=q=2$ 的柯西--施瓦茨不等式,令 $a_i\equiv 1$,$b_i\in\{a,b,c\}$,则
\[
(a+b+c)^2=(1\cdot a+1\cdot b+1\cdot c)^2\le (1^2+1^2+1^2)(a^2+b^2+c^2)=3(a^2+b^2+c^2).
\]

\vspace{0.75em}
\noindent \textbf{例题 3(最值:由凸性得到极值点):}在 $a,b>0$ 且 $a+b=2$ 的条件下,求 $\displaystyle \frac{1}{a}+\frac{1}{b}$ 的最小值。

\noindent \textbf{解:}由例题 1 的 Jensen 结论(或直接用 AM--HM),有
\[
\frac{1}{a}+\frac{1}{b}\ge \frac{4}{a+b}=2.
\]
当且仅当 $a=b=1$ 取等号,因此最小值为 $2$。

\vspace{0.75em}
\noindent \textbf{核心理由:}

\textbf{往年高频考查:}是中学不等式知识的自然高阶延伸,适配现学现考的命题逻辑;

\textbf{衔接性与区分度兼顾:}中学阶段已学习基本不等式、函数单调性,凸/凹函数的定义与相关不等式,既能依托学生已有认知降低入门门槛,又能通过严谨的数学推导考查学生的高阶思维,避免完全陌生知识带来的极端化区分;

\textbf{考查维度多元:}可先给定凸函数定义与不等式形式,再考查例题不等式证明、函数极值求解、结合数列/函数的综合应用,既能考查例题定义理解,也能考查综合迁移能力;

\textbf{贴合院校学科特色:}不等式是工程优化、计算数学的核心工具,与西安交大强势的工科、应用数学学科高度契合。


\subsection*{(二)拓展预测考点(高)}

\subsubsection*{1. 线性代数入门:矩阵基本运算、线性变换、行列式定义与展开法则}
\textbf{详细内容(定义、公式与含义):}

\noindent \textbf{1)矩阵与向量:}一个 $m\times n$ 矩阵是由 $m\cdot n$ 个数排成的表
\[
A=(a_{ij})_{m\times n}.
\]
向量可视为 $n\times 1$ 矩阵 $\bm{x}=(x_1,\dots,x_n)^\mathsf{T}$。

\vspace{0.5em}
\noindent \textbf{2)矩阵基本运算:}
\begin{itemize}
    \item \textbf{加法:}同型矩阵对应元素相加,$(A+B)_{ij}=A_{ij}+B_{ij}$。
    \item \textbf{数乘:}标量 $k$ 乘以矩阵所有元素,$(kA)_{ij}=kA_{ij}$。
    \item \textbf{乘法:}设 $A$ 为 $m\times n$,$B$ 为 $n\times p$,则 $C=AB$ 为 $m\times p$,其元素
    \[
    C_{ij}=\sum_{k=1}^n A_{ik}B_{kj}.
    \]
    \textbf{含义:}第 $i$ 行与第 $j$ 列的“点积”。
\end{itemize}

\vspace{0.5em}
\noindent \textbf{3)线性变换:}映射 $T:\mathbb R^n\to\mathbb R^m$ 若满足
\[
T(\alpha\bm{x}+\beta\bm{y})=\alpha T(\bm{x})+\beta T(\bm{y}),
\]
则称 $T$ 为线性变换。矩阵乘法 $T(\bm{x})=A\bm{x}$ 是最常见的线性变换表示。

\vspace{0.5em}
\noindent \textbf{4)行列式与可逆性:}对 $n\times n$ 矩阵 $A$,其行列式记作 $\det(A)$。
\begin{itemize}
    \item \textbf{几何含义:}线性变换 $\bm{x}\mapsto A\bm{x}$ 对体积的缩放倍数(带符号)。
    \item \textbf{判别:}$A$ 可逆 $\Longleftrightarrow \det(A)\ne 0$。
\end{itemize}

\noindent \textbf{余子式展开(按行/列展开):}对固定行 $i$,
\[
\det(A)=\sum_{j=1}^n a_{ij}A_{ij},\quad A_{ij}=(-1)^{i+j}M_{ij},
\]
其中 $M_{ij}$ 为删去第 $i$ 行第 $j$ 列得到的 $(n-1)\times(n-1)$ 行列式。

\vspace{0.75em}
\noindent \textbf{例题 1(矩阵乘法):}计算
\[
\begin{pmatrix} 1 & 2 \\ 3 & 4 \end{pmatrix}
\begin{pmatrix} 5 & 6 \\ 7 & 8 \end{pmatrix}.
\]

\noindent \textbf{解:}
\[
\begin{pmatrix}
1\cdot 5 + 2\cdot 7 & 1\cdot 6 + 2\cdot 8 \\
3\cdot 5 + 4\cdot 7 & 3\cdot 6 + 4\cdot 8
\end{pmatrix}
=\begin{pmatrix} 19 & 22 \\ 43 & 50 \end{pmatrix}.
\]

\vspace{0.75em}
\noindent \textbf{例题 2(行列式:按行展开):}计算
\[
\begin{vmatrix}
1 & 2 & 3 \\
0 & 4 & 5 \\
1 & 0 & 6
\end{vmatrix}.
\]

\noindent \textbf{解:}按第二行展开:
\[
\det(A)=4\cdot(-1)^{2+2}\begin{vmatrix}1&3\\1&6\end{vmatrix}-5\cdot(-1)^{2+3}\begin{vmatrix}1&2\\1&0\end{vmatrix}
=4(6-3)+5(0-2)=12-10=2.
\]

\vspace{0.75em}
\noindent \textbf{例题 3(线性变换的体积缩放):}设 $A=\begin{pmatrix}2&0\\0&3\end{pmatrix}$,问该变换对平面面积的缩放倍数是多少?

\noindent \textbf{解:}面积缩放倍数为 $|\det(A)|=|2\cdot 3|=6$。

\vspace{0.75em}
\noindent \textbf{核心理由:}

\textbf{学科例题属性:}线性代数是大学所有理工科的必修核心例题,更是西安交大电气、机械、信息类全国顶尖学科的核心数学工具;

\textbf{现学适配性极强:}矩阵的加减乘运算、行列式展开法则,定义清晰、规则明确,可在短材料内完整呈现,能精准考查学生的符号运算、逻辑推导与规则应用能力;

\textbf{命题趋势明确:}去年高中组已涉及线性代数考点,随着初中组命题难度稳步提升,2026年有可能引入例题矩阵、行列式内容,适配分层命题的需求。

\subsubsection*{2. 数值分析初步:拉格朗日/牛顿插值法、数值积分例题}
\textbf{详细内容(定义、公式与含义):}

\noindent \textbf{1)插值问题的定义:}已知函数 $f(x)$ 在 $n+1$ 个互异点 $x_0, x_1, \dots, x_n$ 处的值
$y_0, y_1, \dots, y_n$(其中 $y_i=f(x_i)$),构造一个次数不超过 $n$ 的多项式 $P_n(x)$,使得
\[
P_n(x_i) = y_i,\quad i=0,1,\dots,n.
\]
\textbf{目标:}用一个易计算的多项式在这些节点处“严格吻合”原函数取值。

\vspace{0.5em}
\noindent \textbf{2)拉格朗日(Lagrange)插值:如何构建公式:}

\noindent \textbf{构造思想:}先构造一组“基函数” $L_i(x)$,使其在节点上满足
\[
L_i(x_j)=\delta_{ij}=
\begin{cases}
1,& j=i,\\
0,& j\neq i.
\end{cases}
\]
这样只要令 $P_n(x)=\sum_{i=0}^n y_iL_i(x)$,就会自动满足 $P_n(x_j)=y_j$。

\noindent \textbf{具体构造:}令
\[
L_i(x)=\prod_{j=0, j\ne i}^n \frac{x-x_j}{x_i-x_j}.
\]
\textbf{验证含义:}当 $x=x_j$ 且 $j\neq i$ 时,乘积中包含因子 $(x_j-x_j)=0$,故 $L_i(x_j)=0$;当 $x=x_i$ 时,每个因子为 $(x_i-x_j)/(x_i-x_j)=1$,故 $L_i(x_i)=1$。
因此拉格朗日插值多项式为
\[
P_n(x) = \sum_{i=0}^n y_i L_i(x).
\]

\vspace{0.5em}
\noindent \textbf{3)牛顿(Newton)插值:如何构建公式(差商/插商):}

\noindent \textbf{构造思想:}按节点逐个增加,构造“递推可更新”的形式:
\[
P_n(x)=a_0+a_1(x-x_0)+a_2(x-x_0)(x-x_1)+\dots+a_n\prod_{k=0}^{n-1}(x-x_k).
\]
当新增一个节点时,只需在末尾追加一项(旧系数不变),计算更方便。

\noindent \textbf{系数如何确定:}由插值条件逐个代入节点求系数:
\[
P_n(x_0)=a_0=y_0,\qquad
P_n(x_1)=a_0+a_1(x_1-x_0)=y_1\Rightarrow a_1=\frac{y_1-y_0}{x_1-x_0},
\]
更高阶系数用“差商(Divided Difference)”统一表示。

\noindent \textbf{差商(插商)定义与递推:}
\[
f[x_i]=y_i,
\]
\[
f[x_i,x_{i+1}]=\frac{f[x_{i+1}]-f[x_i]}{x_{i+1}-x_i}=\frac{y_{i+1}-y_i}{x_{i+1}-x_i},
\]
\[
f[x_i,x_{i+1},\dots,x_{i+k}]
=\frac{f[x_{i+1},\dots,x_{i+k}]-f[x_i,\dots,x_{i+k-1}]}{x_{i+k}-x_i}.
\]
\textbf{含义:}差商把“新增节点带来的增量系数”用递推方式算出;牛顿插值的系数恰为
$a_k=f[x_0,\dots,x_k]$。

\noindent \textbf{牛顿插值公式:}
\[
P_n(x)= f[x_0] + f[x_0, x_1](x-x_0) + \dots + f[x_0, \dots, x_n](x-x_0)\dots(x-x_{n-1}).
\]
\textbf{含义:}每增加一个点只需在末尾增加一项,计算更具递推性与增量更新特征。

\vspace{0.5em}
\noindent \textbf{4)数值积分(梯形法则):}
\[
\int_a^b f(x) dx \approx \frac{b-a}{2} [f(a) + f(b)].
\]

\vspace{0.75em}
\noindent \textbf{例题 1(Lagrange 插值):}给定 $(x_0,y_0)=(0,1), (x_1,y_1)=(1,2), (x_2,y_2)=(2,0)$,求二次插值多项式 $P_2(x)$。

\noindent \textbf{解:}
基函数:$L_0(x) = \frac{(x-1)(x-2)}{(0-1)(0-2)} = \frac{1}{2}(x^2-3x+2)$;
$L_1(x) = \frac{(x-0)(x-2)}{(1-0)(1-2)} = -x(x-2)$;
$L_2(x) = \frac{(x-0)(x-1)}{(2-0)(2-1)} = \frac{1}{2}x(x-1)$。
$P_2(x) = 1\cdot L_0(x) + 2\cdot L_1(x) + 0\cdot L_2(x) = \frac{1}{2}(x^2-3x+2) - 2x^2 + 4x = -\frac{3}{2}x^2 + \frac{5}{2}x + 1$。

\vspace{0.75em}
\noindent \textbf{例题 2(数值积分):}用梯形公式估计 $\int_0^1 x^2 dx$ 的值。

\noindent \textbf{解:}$a=0, b=1, f(x)=x^2$。
$\int_0^1 x^2 dx \approx \frac{1-0}{2} [f(0) + f(1)] = \frac{1}{2} [0^2 + 1^2] = 0.5$。(注:精确值为 $1/3 \approx 0.33$)

\vspace{0.75em}
\noindent \textbf{核心理由:}

\textbf{往年命题铺垫:}往年已考查过牛顿插值法相关内容,是西安交大计算数学、工科计算领域的核心例题;

\textbf{适配现学现考模式:}插值法是中学函数知识的自然延伸,给定插值公式定义与推导逻辑后,可考查学生的代数运算、归纳推理、公式迁移能力,同时可结合工程实际场景出题,贴合西安交大的工科强势定位;

\textbf{选拔价值突出:}数值分析是大学工科将数学理论转化为实际问题解决方案的核心工具,能精准考查学生的数学应用能力,而非纯理论推导能力。

\subsubsection*{3. 离散数学初步:同余与模运算进阶、图论例题、组合数学生成函数}
\textbf{详细内容(定义、公式与含义):}

\noindent \textbf{1)同余与模运算:}若 $a-b$ 能被 $m$ 整除,则称 $a, b$ 对模 $m$ 同余,记作 $a \equiv b \pmod{m}$。
\begin{itemize}
    \item \textbf{费马小定理:}若 $p$ 为质数,且 $\gcd(a, p)=1$,则 $a^{p-1} \equiv 1 \pmod{p}$。
    \item \textbf{欧拉定理:}若 $\gcd(a, m)=1$,则 $a^{\phi(m)} \equiv 1 \pmod{m}$,其中 $\phi(m)$ 是欧拉函数。
\end{itemize}

\noindent \textbf{2)图论例题:}图 $G=(V, E)$ 由顶点集 $V$ 和边集 $E$ 组成。
\begin{itemize}
    \item \textbf{握手定理:}所有顶点的度数之和等于边数的两倍:$\sum_{v\in V} \text{deg}(v) = 2|E|$。
    \item \textbf{欧拉公式(平面图):}$V - E + F = 2$,其中 $F$ 为面数。
\end{itemize}

\noindent \textbf{3)生成函数(母函数):}数列 $\{a_n\}$ 的普通生成函数为 $G(x) = \sum_{n=0}^\infty a_n x^n$。
\textbf{含义:}将组合问题的计数转化为多项式的系数运算。

\vspace{0.75em}
\noindent \textbf{例题 1(同余计算):}求 $3^{2026}$ 除以 $7$ 的余数。

\noindent \textbf{解:}由费马小定理,因 $7$ 是质数且 $\gcd(3,7)=1$,有 $3^6 \equiv 1 \pmod{7}$。
$2026 = 6 \times 337 + 4$。
故 $3^{2026} = (3^6)^{337} \cdot 3^4 \equiv 1^{337} \cdot 81 \pmod{7}$。
$81 = 7 \times 11 + 4$,故余数为 $4$。

\vspace{0.75em}
\noindent \textbf{例题 2(图论):}一个平面图有 10 个顶点,每个顶点的度数均为 3,求该图将平面分成了多少个区域?

\noindent \textbf{解:}由握手定理,$2E = \sum \text{deg}(v) = 10 \times 3 = 30 \Rightarrow E = 15$。
由欧拉公式 $V - E + F = 2 \Rightarrow 10 - 15 + F = 2 \Rightarrow F = 7$。
区域数(含外部无限面)为 7。

\vspace{0.75em}
\noindent \textbf{核心理由:}

\textbf{往年考情支撑:}往年已考查过同余定理相关证明,是竞赛与大学离散数学的入门核心内容;

\textbf{现学适配性高:}模运算、图论的顶点/边/连通性等例题定义,规则简单、无大量前置知识要求,可快速理解掌握,能精准考查学生的逻辑推理、组合思维与数学建模能力;

\textbf{学科趋势契合:}随着人工智能、计算机科学的快速发展,离散数学的重要性持续提升,与西安交大计算机、人工智能强势学科高度契合,是近年命题的重要延伸方向。


\section*{二、创新设计(物理)考点预测}

\subsection*{(一)核心高频考点(大)}

\subsubsection*{1. 狭义相对论例题:洛伦兹变换、时空效应、相对论动量与能量}
\textbf{详细内容(定义、公式与含义):}

\noindent \textbf{1)洛伦兹变换:}设两个惯性系 $S$ 和 $S'$,其中 $S'$ 相对于 $S$ 以速度 $v$ 沿 $x$ 轴正方向运动。变换关系为:
\[
x' = \gamma(x - vt), \quad y' = y, \quad z' = z, \quad t' = \gamma\left(t - \frac{vx}{c^2}\right)
\]
其中 $\gamma = \frac{1}{\sqrt{1 - v^2/c^2}}$ 称为洛伦兹因子。

\noindent \textbf{2)时空效应:}
\begin{itemize}
    \item \textbf{时间膨胀(动钟变慢):}$\Delta t = \gamma \Delta \tau$,其中 $\Delta \tau$ 是原时(静止系时间)。
    \item \textbf{长度收缩(动尺变短):}$L = \frac{L_0}{\gamma}$,其中 $L_0$ 是原长。
\end{itemize}

\noindent \textbf{3)相对论动力学:}
\begin{itemize}
    \item \textbf{相对论质量:}$m = \gamma m_0$,其中 $m_0$ 为静质量。
    \item \textbf{质能方程:}$E = mc^2 = \gamma m_0 c^2$。
    \item \textbf{动量:}$p = mv = \gamma m_0 v$。
\end{itemize}

\vspace{0.75em}
\noindent \textbf{例题 1(时间膨胀):}一束 $\mu$ 子以 $0.99c$ 的速度运动。已知 $\mu$ 子在静止状态下的平均寿命为 $2.2 \times 10^{-6}$ s。求实验室内测得的 $\mu$ 子寿命。

\noindent \textbf{解:}洛伦兹因子 $\gamma = \frac{1}{\sqrt{1 - 0.99^2}} \approx 7.09$。
实验室测得寿命 $\Delta t = \gamma \Delta \tau = 7.09 \times 2.2 \times 10^{-6} \text{ s} \approx 1.56 \times 10^{-5} \text{ s}$。

\vspace{0.75em}
\noindent \textbf{例题 2(质能转换):}计算将 $1 \text{ g}$ 物质完全转化为能量时释放的热量。

\noindent \textbf{解:}$m = 10^{-3} \text{ kg}$,$c = 3 \times 10^8 \text{ m/s}$。
$E = mc^2 = 10^{-3} \times (3 \times 10^8)^2 = 9 \times 10^{13} \text{ J}$。

\vspace{0.75em}
\noindent \textbf{核心理由:}历年经典核心考点:过往多年多次考查相对论相关内容,2025年创新设计模块全部为大学物理内容,相对论是核心考查点,是西少复试现学现考的标志性题型;

\textbf{选拔属性极强:}中学物理仅涉及经典力学时空观,相对论完全颠覆了学生的固有认知,需要学生快速接受全新的时空定义与物理规则,能彻底规避刷题带来的优势,100\%聚焦学生的新理论接受能力、逻辑推导与模型构建能力;

\textbf{出题空间充足:}可从例题的洛伦兹变换公式应用,进阶到时间膨胀、长度收缩的定量计算,再到相对论动量与能量的综合问题,层层递进的难度完全适配大题考查需求;学科例题价值:是近代物理的核心基石,契合少年班拔尖物理人才的培养目标。

\subsubsection*{2. 量子力学入门:玻尔氢原子模型、光电效应、德布罗意物质波、波函数例题}
\textbf{详细内容(定义、公式与含义):}

\noindent \textbf{1)光电效应:}入射光照射金属时会有电子逸出。
\begin{itemize}
    \item \textbf{爱因斯坦光电方程:}$h\nu = \phi + K_{\max}$。
    \item \textbf{含义:}$h$ 为普朗克常数,$\nu$ 为频率;$\phi$ 为逸出功;$K_{\max}=\tfrac12 m_e v_{\max}^2$ 为逸出电子最大动能。
    \item \textbf{截止频率:}$\nu_0 = \phi/h$(低于该频率不发生光电效应)。
\end{itemize}

\noindent \textbf{2)德布罗意物质波:}微观粒子具有波动性,其波长
\[
\lambda = \frac{h}{p}.
\]
\textbf{含义:}动量越大,波长越短,宏观物体波长极小所以波动性不明显。

\vspace{0.5em}
\noindent \textbf{3)玻尔氢原子模型:}
\begin{itemize}
    \item \textbf{角动量量子化:}$m_e v r = n\hbar,\ n=1,2,\dots$。
    \item \textbf{能级:}$E_n = -\frac{13.6\ \text{eV}}{n^2}$。
    \item \textbf{跃迁与谱线:}$h\nu = |E_{n_2}-E_{n_1}|$。
\end{itemize}

\vspace{0.5em}
\noindent \textbf{4)波函数与概率解释:}用波函数 $\psi(\bm r,t)$ 描述体系状态。
\begin{itemize}
    \item \textbf{Born 解释:}$|\psi|^2$ 给出概率密度。
    \item \textbf{归一化:}$\int |\psi|^2\, d\tau = 1$。
\end{itemize}

\vspace{0.75em}
\noindent \textbf{例题 1(光电效应):}已知某金属逸出功 $\phi=2.0\ \text{eV}$,入射光子能量为 $3.5\ \text{eV}$,求逸出电子最大动能(单位 eV)。

\noindent \textbf{解:}由 $h\nu=\phi+K_{\max}$,得 $K_{\max}=3.5-2.0=1.5\ \text{eV}$。

\vspace{0.75em}
\noindent \textbf{例题 2(玻尔能级跃迁):}氢原子从 $n=3$ 跃迁到 $n=2$,求辐射光子的能量(单位 eV)。

\noindent \textbf{解:}$E_3=-13.6/9\ \text{eV}$,$E_2=-13.6/4\ \text{eV}$。
故 $\Delta E = |E_2-E_3| = 13.6\left(\frac{1}{4}-\frac{1}{9}\right)=13.6\cdot\frac{5}{36}\approx 1.89\ \text{eV}$。

\vspace{0.75em}
\noindent \textbf{核心理由:}连续多年核心考查:光电效应、德布罗意波、玻尔模型的定量计算,是近代物理模块的核心命题方向;

\textbf{完美适配现学现考:}中学物理仅涉及量子力学的科普级内容,严格的定量计算与理论定义是大学物理专属内容,可在短材料内清晰呈现核心公式与物理意义,精准考查学生的模型构建、公式迁移与定量计算能力;

\textbf{贴合院校学科特色:}西安交大核科学与技术、物理学科实力强劲,量子力学是这些学科的核心例题,命题高度贴合学校的王牌学科优势;

\textbf{创新考查适配性:}可结合实验设计、现象解释、定量计算综合出题,既考查物理思维,也考查创新设计能力,完全贴合科目定位。

\subsubsection*{3. 电磁学进阶:麦克斯韦方程组例题、电磁场统一理论、电磁辐射与电磁波}
\textbf{详细内容(定义、公式与含义):}

\noindent \textbf{1)麦克斯韦方程组(微分形式):}
\begin{align*}
\nabla\cdot \bm{E} &= \frac{\rho}{\varepsilon_0} \quad &\text{(高斯定律:电荷是电场源)}\\
\nabla\cdot \bm{B} &= 0 \quad &\text{(无磁单极子)}\\
\nabla\times \bm{E} &= -\frac{\partial \bm{B}}{\partial t} \quad &\text{(法拉第电磁感应)}\\
\nabla\times \bm{B} &= \mu_0 \bm{J} + \mu_0\varepsilon_0\frac{\partial \bm{E}}{\partial t} \quad &\text{(安培-麦克斯韦:含位移电流)}
\end{align*}

\noindent \textbf{含义:}四个方程把电场 $\bm{E}$、磁场 $\bm{B}$ 与电荷密度 $\rho$、电流密度 $\bm{J}$ 联系起来,是经典电磁学的统一描述。

\vspace{0.5em}
\noindent \textbf{2)电磁波:}在真空中($\rho=0,\bm{J}=0$),可推出电场与磁场满足波动方程,其传播速度
\[
 c=\frac{1}{\sqrt{\mu_0\varepsilon_0}}.
\]

\vspace{0.5em}
\noindent \textbf{3)洛伦兹力:}带电粒子在电磁场中受力
\[
\bm{F}=q(\bm{E}+\bm{v}\times\bm{B}).
\]
\textbf{含义:}电场对电荷做功改变速率,磁场力与速度垂直只改变方向。

\vspace{0.75em}
\noindent \textbf{例题 1(电磁波速度):}已知真空介电常数 $\varepsilon_0=8.85\times 10^{-12}\ \text{F/m}$,真空磁导率 $\mu_0=4\pi\times 10^{-7}\ \text{H/m}$,求电磁波在真空中的传播速度 $c$。

\noindent \textbf{解:}
\[
 c=\frac{1}{\sqrt{\mu_0\varepsilon_0}}\approx \frac{1}{\sqrt{(4\pi\times 10^{-7})(8.85\times 10^{-12})}}\approx 3.0\times 10^8\ \text{m/s}.
\]

\vspace{0.75em}
\noindent \textbf{例题 2(洛伦兹力与圆周运动):}一个电荷量为 $q$、质量为 $m$ 的粒子以速度 $v$ 垂直进入匀强磁场 $\bm{B}$,求其圆周运动半径。

\noindent \textbf{解:}磁场力大小 $F=qvB$,作为向心力 $mv^2/r$,故
\[
qvB=\frac{mv^2}{r}\Rightarrow r=\frac{mv}{qB}.
\]

\vspace{0.75em}
\noindent \textbf{核心理由:}近年高频考查方向:辐射压强、电磁波、电磁场统一理论、洛伦兹力三维分析,是电磁学模块的核心区分度题型;

\textbf{衔接性与高阶性兼顾:}中学物理已学习电场、磁场的基本定律,麦克斯韦方程组是电磁学的集大成者,既是中学知识的高阶延伸,又能考查学生对“场”的思维体系的快速构建能力,适配现学现考模式;

\textbf{院校学科适配性拉满:}西安交大电气工程学科全国顶尖,电磁学是电气学科的核心例题,是命题最贴合学校王牌专业的核心方向;出题灵活性强:可从例题的电场磁场通量、环量计算,进阶到麦克斯韦方程组的简单应用,再到电磁波传播、辐射压的定量计算,可结合工程实际场景出题,考查学生的创新应用能力。


\subsection*{(二)拓展预测考点(高)}

\subsubsection*{1. 热力学与统计物理例题:热力学第二定律、熵的定义与统计意义、卡诺循环进阶}
\textbf{详细内容(定义、公式与含义):}

\noindent \textbf{1)热力学第二定律:}描述自然过程的方向性。
\begin{itemize}
    \item \textbf{克劳修斯表述:}热量不能自发地从低温物体传到高温物体。
    \item \textbf{开尔文表述:}不可能从单一热源吸热并把它完全变为功而不产生其他影响。
    \item \textbf{数学表述:}对任意循环,有 $\oint \frac{\delta Q}{T} \le 0$(等号对应可逆循环)。
\end{itemize}

\noindent \textbf{2)熵(Entropy):}
\begin{itemize}
    \item \textbf{定义:}对可逆过程,$dS = \frac{\delta Q_{\text{rev}}}{T}$。
    \item \textbf{统计意义:}$S = k_B \ln \Omega$,其中 $\Omega$ 为系统微观状态数。
    \item \textbf{熵增原理:}孤立系统的熵不减,$\Delta S \ge 0$。
\end{itemize}

\noindent \textbf{3)卡诺循环与热机效率:}
\begin{itemize}
    \item \textbf{卡诺效率:}$\eta = 1 - \frac{T_C}{T_H}$,其中 $T_H$、$T_C$ 分别为高温与低温热源温度(开尔文温标)。
    \item \textbf{含义:}在给定两热源温度下,任何热机效率都不可能超过卡诺效率。
\end{itemize}

\vspace{0.75em}
\noindent \textbf{例题 1(熵变:等温膨胀):}理想气体在温度 $T$ 下做准静态等温膨胀,体积从 $V_1$ 变为 $V_2$,求熵变。

\noindent \textbf{解:}等温过程有 $\Delta S = \frac{Q_{\text{rev}}}{T}$。而 $Q_{\text{rev}} = nRT\ln\frac{V_2}{V_1}$。
因此
\[
\Delta S = nR\ln\frac{V_2}{V_1}.
\]

\vspace{0.75em}
\noindent \textbf{例题 2(卡诺热机效率):}高温热源 $T_H=600\ \text{K}$,低温热源 $T_C=300\ \text{K}$,求卡诺热机最大效率。

\noindent \textbf{解:}$\eta = 1 - \frac{T_C}{T_H} = 1 - \frac{300}{600} = 0.5$,即 $50\%$。

\vspace{0.75em}
\noindent \textbf{核心理由:}近年命题热点铺垫:2025年面试考查热力学与生命科学综合题,笔试也涉及热力学相关内容,是近年的命题热点方向;

\textbf{现学适配性高:}中学物理仅学习热力学第一定律,热力学第二定律、熵的概念是大学热力学的核心,定义清晰、物理意义明确,可在短材料内完整呈现,能考查学生宏观微观的思维转化能力与逻辑推理能力;

\textbf{贴合院校学科优势:}西安交大能源与动力工程学科全国顶尖,热力学是该学科的核心例题,命题高度贴合学校的学科特色;

\textbf{跨学科创新属性:}熵的概念可跨学科延伸到生命科学、信息科学,能考查学生的跨学科迁移能力,契合创新设计的考查目标。

\subsubsection*{2. 分析力学入门:虚功原理、拉格朗日力学例题}
\textbf{详细内容(定义、公式与含义):}

\noindent \textbf{1)约束与广义坐标:}对有约束系统,可用最少个独立变量 $q_1,\dots,q_s$(广义坐标)描述系统状态,速度为 $\dot q_i$。

\vspace{0.5em}
\noindent \textbf{2)虚位移与虚功:}
\begin{itemize}
    \item \textbf{虚位移:}$\delta \bm r$ 是在同一时刻、满足约束条件的“假想微小位移”。
    \item \textbf{虚功:}$\delta W = \sum_i \bm{F}_i\cdot \delta \bm r_i$。
\end{itemize}

\noindent \textbf{虚功原理(静力平衡):}若系统处于静力平衡且约束力不做虚功(理想约束),则对任意允许的虚位移有
\[
\delta W = 0.
\]

\vspace{0.5em}
\noindent \textbf{3)拉格朗日函数与方程:}
\begin{itemize}
    \item \textbf{拉格朗日量:}$L = T - V$,其中 $T$ 为动能,$V$ 为势能。
    \item \textbf{拉格朗日方程:}$\displaystyle \frac{d}{dt}\left(\frac{\partial L}{\partial \dot q_i}\right) - \frac{\partial L}{\partial q_i} = 0\quad (i=1,\dots,s)$。
    \item \textbf{含义:}用能量形式统一写出运动方程,避免逐个分解力的矢量分量。
\end{itemize}

\vspace{0.75em}
\noindent \textbf{例题 1(虚功原理求平衡):}轻绳绕过光滑定滑轮,两端分别悬挂质量为 $m_1$ 与 $m_2$ 的物体,系统静止。求两端绳中的张力 $T$,并写出平衡条件。

\noindent \textbf{解:}对 $m_1$:受力平衡 $T - m_1 g = 0$;对 $m_2$:$T - m_2 g = 0$。
若系统静止,则必须同时满足 $m_1=m_2$,此时 $T=m_1 g=m_2 g$。
(等价地,取虚位移 $\delta x$,两物体位移大小相等方向相反,则 $\delta W=(m_1 g - m_2 g)\delta x=0\Rightarrow m_1=m_2$。)

\vspace{0.75em}
\noindent \textbf{例题 2(拉格朗日方程:单摆小角度):}长度为 $l$ 的单摆,取广义坐标为摆角 $\theta$。在小角度近似下推导运动方程。

\noindent \textbf{解:}动能 $T=\tfrac12 m(l\dot\theta)^2=\tfrac12 ml^2\dot\theta^2$。
势能(取最低点为零)$V= mgl(1-\cos\theta)\approx \tfrac12 mgl\theta^2$。
故 $L=T-V=\tfrac12 ml^2\dot\theta^2-\tfrac12 mgl\theta^2$。
代入拉格朗日方程:
\[
\frac{d}{dt}(ml^2\dot\theta)+mgl\theta=0\Rightarrow \ddot\theta+\frac{g}{l}\theta=0.
\]

\vspace{0.75em}
\noindent \textbf{核心理由:}极致的现学现考适配性:中学物理学习的是牛顿力学的矢量分析体系,拉格朗日力学是基于能量、变分的全新力学体系,对所有考生而言几乎是完全陌生的内容,能绝对公平地考查学生的新理论接受能力、建模与应用能力,彻底规避提前学习的优势;

\textbf{考查目标高度契合:}可先给定虚功原理、拉格朗日方程的定义与简单推导,再考查力学系统建模、平衡条件计算、运动方程推导,精准考查学生的物理建模能力,完全贴合创新设计的核心考查目标;

\textbf{学科例题价值:}是大学物理、机械、航空航天等学科的核心例题,契合少年班的工科培养定位。

\subsubsection*{3. 波动光学进阶:光的干涉衍射定量计算、傅里叶光学例题、偏振光学}
\textbf{详细内容(定义、公式与含义):}

\noindent \textbf{1)光的干涉(双缝干涉):}两束相干光叠加形成稳定条纹。
\begin{itemize}
    \item \textbf{相位差与光程差:}$\Delta \phi = \frac{2\pi}{\lambda}\Delta L$。
    \item \textbf{明纹条件:}$\Delta L = n\lambda$;\textbf{暗纹条件:}$\Delta L = (n+\frac12)\lambda$。
    \item \textbf{条纹间距:}$\Delta x = \frac{D}{d}\lambda$($D$ 为缝到屏距离,$d$ 为双缝间距)。
\end{itemize}

\noindent \textbf{2)光的衍射(单缝衍射):}波绕过狭缝后在远处形成衍射图样。
\begin{itemize}
    \item \textbf{暗纹条件:}$a\sin\theta = n\lambda\ (n=1,2,\dots)$,其中 $a$ 为缝宽。
    \item \textbf{中央明纹角宽:}$\Delta\theta \approx \frac{2\lambda}{a}$(小角近似)。
\end{itemize}

\noindent \textbf{3)偏振光学:}描述电场矢量的振动方向。
\begin{itemize}
    \item \textbf{马吕斯定律:}$I = I_0\cos^2\alpha$。
    \item \textbf{布儒斯特角:}$\tan i_p = \frac{n_2}{n_1}$。
\end{itemize}

\vspace{0.75em}
\noindent \textbf{例题 1(双缝条纹间距):}波长 $\lambda=600\ \text{nm}$,双缝间距 $d=0.10\ \text{mm}$,屏距 $D=1.0\ \text{m}$,求相邻明纹间距。

\noindent \textbf{解:}$\Delta x = \frac{D}{d}\lambda = \frac{1}{1.0\times10^{-4}}\cdot 600\times10^{-9} = 6.0\times10^{-3}\ \text{m}=6\ \text{mm}$。

\vspace{0.75em}
\noindent \textbf{例题 2(马吕斯定律):}两偏振片透振方向夹角为 $60^\circ$,一束强度为 $I_0$ 的线偏振光通过第一片后入射第二片,求透射强度。

\noindent \textbf{解:}$I = I_0 \cos^2 60^\circ = I_0\cdot (1/2)^2 = I_0/4$。

\vspace{0.75em}
\noindent \textbf{核心理由:}往年考情支撑:光的衍射效应,是波动光学模块的高频考点;

\textbf{现学现考适配性强:}中学物理仅涉及波动光学的定性内容,定量的干涉衍射计算、傅里叶光学是大学物理专属内容,定义清晰、公式与物理模型明确,可在短材料内完整呈现;

\textbf{创新设计属性突出:}可结合光学仪器设计、成像原理、光谱分析等创新类题目,既考查物理知识的迁移应用,也考查学生的工程创新设计能力,完美贴合科目名称;

\textbf{学科贴合度高:}与西安交大精密仪器强势学科高度契合,是命题的重要延伸方向。


\section*{三、2026年复试笔试整体命题趋势补充}

大概率延续2025年的题型设置,主观大题,侧重推导过程、逻辑严谨性,而非最终答案的正确性;延续书面文字材料的现学现考形式,对学生的文本阅读、抽象概念快速理解能力要求进一步提升;初中组与高中组继续分开命题、分开排序,初中组侧重中学知识的大学高阶延伸,高中组以大学专业例题知识点为核心;所有题目均不会考查需要大量记忆的内容,核心聚焦“现学现用”能力,重点考查定义理解、逻辑推导、知识迁移、创新应用四大核心素养,而非学生提前学习大学课程的体量。

\textbf{总结:}2026年西安交大少年班复试预计将继续保持"现学现考"模式,数学侧重分析例题(极限、实数理论)和代数结构(群论),物理侧重相对论和量子力学两大现代物理支柱,同时兼顾电磁学和振动波动经典内容。考生应在巩固高中例题的同时,提前接触大学例题课程,培养快速学习和知识迁移能力。


\newpage
% =========================================================
% 第二份文档:面试指南
% =========================================================
\section*{西交大少年班面试特别指南}

\textbf{1. 遇到不懂的问题怎么办?}

\noindent 不要乱编。可以这样回答:“这个具体的知识点我目前没有涉猎,但我可以根据我学过的知识尝试分析一下...” —— \textbf{展示思维过程比展示结果更重要。}

\vspace{1em}
\textbf{2. 如何体现“少年班”特质?}

\noindent 少年班寻找的是“早慧”且“潜力大”的学生。回答时要做到:\textbf{观点清晰}、\textbf{逻辑完整}、\textbf{敢于提出假设并验证}。遇到开放题可用“现象—原因—影响—对策”的结构组织语言。

\vspace{1em}
\textbf{3. 面试题库(2025-2026全球热点,题型对齐往年出题思路):}

\vspace{0.5em}
\noindent \textbf{### 人文案例(14题)}

\noindent \textbf{1)诗句理解与迁移:}
“少年心事当拏云,谁念幽寒坐呜呃。”请谈谈你对这句话的理解,并引用(或自拟)一句类似风格的句子表达“逆境中的志气”。

\vspace{0.5em}
\noindent \textbf{2)典籍出处与观点阐释:}
“仓廪实而知礼节,衣食足而知荣辱。”出自哪里?请解释含义,并联系当下“消费主义/网络攀比”谈你的看法。

\vspace{0.5em}
\noindent \textbf{3)历史情境代入(选择与代价):}
假设你身处一次重大历史转折(例如变法、和议、迁都、改革),必须在“短期稳定”和“长期改革”之间做选择。你会如何决策?你的判断标准是什么?

\vspace{0.5em}
\noindent \textbf{4)信息时代的“道听途说”:}
你在短视频平台看到“某地发生重大事件”的爆料,但来源不明。你会如何判断真假?请给出至少三步验证方法,并说明每一步的意义。

\vspace{0.5em}
\noindent \textbf{5)公共表达与边界:}
有人在网上发表激烈观点引发争议。你认为“表达自由”和“伤害他人/煽动对立”的边界在哪里?请举一个现实情境说明。

\vspace{0.5em}
\noindent \textbf{6)伦理题(AI与隐私):}
为了提升公共安全,在城市中大规模部署摄像头与算法识别是否合理?你支持或反对的条件分别是什么?

\vspace{0.5em}
\noindent \textbf{7)价值题(善意与激励):}
学校设立“志愿服务学分/奖学金加分”。你认为这会促进善意还是会让善意功利化?如何设计规则才能兼顾公平与激励?

\vspace{0.5em}
\noindent \textbf{8)图像寓意题(文字描述替代图片):}
一幅海报:上半部分是不断扩建的数据中心,电表飞转;下半部分是干涸的河床与龟裂土地,中间写着“AI Boom”。问:海报想表达什么?你如何平衡科技发展与环境代价?

\vspace{0.5em}
\noindent \textbf{9)文化商业化与“爆款逻辑”:}
某地为了文旅爆火,大量复制“同款古风街区”,结果千城一面。你如何理解“文化传承、创新、商业化”三者关系?如何避免同质化?

\vspace{0.5em}
\noindent \textbf{10)翻译并理解(自我与他者):}
翻译并谈理解:\\
\textit{Your online self is a story you keep editing; your real self is the one who still needs courage when the screen goes dark.}

\vspace{0.5em}
\noindent \textbf{11)审美表达(书法/艺术与人格):}
给你一幅行草作品(或描述其线条快慢、轻重、疏密变化),请谈:你从中读到怎样的情绪与气质?这与“学习中的心态管理”有什么相通之处?

\vspace{0.5em}
\noindent \textbf{12)文化创新案例(AI参与创作):}
如果用AI生成国风音乐、国画风格海报、短剧脚本,这算“创新”还是“偷懒”?你如何看待署名、版权、原创性?

\vspace{0.5em}
\noindent \textbf{13)科技竞争与精神:}
以“开源大模型生态/国产算力/工程化突破”为背景,谈谈你最佩服的一种精神品质是什么(如长期主义、协作、复盘、敢于试错等),并举一个你自己的例子说明。

\vspace{0.5em}
\noindent \textbf{14)国际关系与青年责任:}
当国际环境紧张、技术封锁加剧时,青年应更“竞争”还是更“合作”?你认为更理想的姿态是什么?为什么?

\vspace{1em}
\noindent \textbf{### 科学案例(7题)}

\noindent \textbf{1)低空经济与飞行原理:}
固定翼飞机、直升机、多旋翼无人机、eVTOL(倾转旋翼/倾转翼)产生升力的方式有什么共同点和差异?请用“动量变化/压力差/能量”来解释。

\vspace{0.5em}
\noindent \textbf{2)热传导与生存概率:}
同样温度下,人在风中与在水中哪个更容易失温?为什么?如果给你材料(铝箔毯、棉衣、塑料雨衣),你会如何组合以延长保温时间?

\vspace{0.5em}
\noindent \textbf{3)设计实验(电池安全):}
请设计一个实验,验证“温度、外力挤压、过充”对锂电池安全的影响。要求说明:对照组、变量、如何观察结果、如何保证安全。

\vspace{0.5em}
\noindent \textbf{4)高速交通与空气动力:}
高铁速度提升会遇到哪些主要阻力?阻力与速度大致是什么关系?如果要进一步提速,工程上可能从哪些方向优化?

\vspace{0.5em}
\noindent \textbf{5)医学成像与物理机制:}
列举两种医学成像(CT/MRI/超声/PET/内窥镜等),说明各自利用的物理原理,并比较它们的优缺点(分辨率、辐射、成本、适用场景)。

\vspace{0.5em}
\noindent \textbf{6)生命存在条件(结合近年天文热点):}
如果在系外行星上寻找生命迹象,你会优先看哪些指标(液态水、温度范围、大气成分、磁场、恒星活动等)?请说明理由与可能的探测方法。

\vspace{0.5em}
\noindent \textbf{7)传感器与自动驾驶/机器人:}
比较激光雷达、毫米波雷达、摄像头三者的优缺点。若在雨雾/夜间/强逆光环境下,哪种更可靠?为什么需要多传感器融合?

\vspace{1em}
\noindent \textbf{### 小组辩论(8题)}

\noindent \textbf{1)AI辅助学习:}中学生使用生成式AI完成作业,利大于弊还是弊大于利?\\
\textbf{2)开源与安全:}强大AI模型应该尽量开源,还是应该严格限制?\\
\textbf{3)科技与能耗:}为了AI与算力发展增加用电与碳排,是否合理?\\
\textbf{4)未成年人网络保护:}限制短视频/游戏时长,对成长利大于弊还是弊大于利?\\
\textbf{5)科研竞争:}在国际竞争下,科研应优先“速度”还是优先“可靠性与可复现”?\\
\textbf{6)自动驾驶责任:}自动驾驶出事故,责任应更多由厂商承担还是由使用者承担?\\
\textbf{7)动物伦理:}为了医学/公共健康牺牲动物生命是否可取?\\
\textbf{8)外星探索风险:}人类探索外星更可能带来机遇还是灾难?(行星保护、污染风险、未知风险)

\vspace{1em}
\noindent \textbf{### 小心理测试(7题)}

\noindent \textbf{1)压力情境:}如果你在面试中突然卡壳,你会怎么做来恢复思路?\\
\textbf{2)失败经验:}说一次你认真准备却没达到预期的经历。你如何复盘?下次会怎么改?\\
\textbf{3)自我评价:}你最想保留的一个优点是什么?最想改变的一个习惯是什么?为什么?\\
\textbf{4)人际冲突:}你和同学/老师意见不一致时,通常会坚持、妥协还是沟通?举例说明。\\
\textbf{5)长期目标:}你做过最长时间坚持的一件事是什么?靠什么坚持下来?\\
\textbf{6)诱惑与自律:}当手机/游戏影响学习时,你会用什么方法管理自己?\\
\textbf{7)同理心问题:}如果你发现队友在小组任务中明显拖后腿,你会怎么处理,既推进任务又不伤害关系?

\vspace{1em}
\noindent \textbf{### 小组讨论(7题)}

\noindent \textbf{1)自我介绍追问:}你介绍里提到的一个兴趣/项目,最难的点是什么?你具体怎么解决?\\
\textbf{2)观点形成:}你最近一次改变自己观点的经历是什么?是什么证据/逻辑让你改变?\\
\textbf{3)合作分工:}团队里你更像组织者、执行者还是创意者?为什么?\\
\textbf{4)时间管理:}如果一天只有2小时自由学习时间,你会如何安排,并说明依据?\\
\textbf{5)公共议题:}给一个校园议题(例如AI使用规范、手机管理),请提出一条可执行的规则并解释利弊。\\
\textbf{6)问题拆解:}现场给一个陌生问题,你如何快速拆解成可回答的结构?请现场示范。\\
\textbf{7)反思与成长:}你希望大学四年获得的最核心能力是什么?你打算怎么培养?

\vspace{1em}
\textbf{4. 必背物理常识(结合热点):}
\begin{itemize}
    \item \textbf{能量守恒:}所有永动机(包括某些号称无需充电的AI设备)都是骗局。
    \item \textbf{波粒二象性:}量子现象与信息技术的基础背景。
    \item \textbf{熵增原理:}系统趋于无序;信息处理与能耗、散热问题常可联系“熵”的概念进行解释。
\end{itemize}
\section*{西交大少年班面试特别指南2}

\textbf{1. 遇到不懂的问题怎么办?}

\noindent 不要乱编。可以这样回答:“这个具体的知识点我目前没有涉猎,但我可以根据我学过的知识尝试分析一下...” —— \textbf{展示思维过程比展示结果更重要。}

\vspace{1em}
\textbf{2. 如何体现“少年班”特质?}

\noindent 少年班寻找的是“早慧”且“潜力大”的学生。回答时要做到:\textbf{观点清晰}、\textbf{逻辑完整}、\textbf{敢于提出假设并验证}。遇到开放题可用“现象—原因—影响—对策”的结构组织语言。

\vspace{1em}
\textbf{3. 面试题库(版本二:社会议题与科技伦理,2025-2026全球热点):}

\vspace{0.5em}
\noindent \textbf{### 人文案例(14题)}

\noindent \textbf{1)诗句与时代情绪:}
“长风破浪会有时,直挂云帆济沧海。”在不确定性越来越大的时代(科技加速、行业变化),你如何理解“长风破浪”?你会如何规划自己的学习与风险?

\vspace{0.5em}
\noindent \textbf{2)典籍与责任:}
“天下兴亡,匹夫有责。”这句话在今天还适用吗?面对环境保护、网络暴力、技术滥用等议题,中学生的“责”体现在哪里?

\vspace{0.5em}
\noindent \textbf{3)选择题:安稳 vs 担当:}
如果你可以选择一个“安稳但平凡”的人生,或一个“常伴风险但可能影响社会”的人生,你会选哪一个?请从家庭、社会、个人成长三个角度分析。

\vspace{0.5em}
\noindent \textbf{4)信息茧房与批判性思维:}
推荐算法会给我们推送“越看越像”的内容。你觉得这对青少年有什么好处和风险?你有什么办法“跳出信息茧房”?

\vspace{0.5em}
\noindent \textbf{5)公共空间的文明边界:}
有人在地铁上大声刷短视频,认为“我没违法,别管我”。你怎么看待“法律底线”和“公共文明”的关系?如果你现场看到,会怎么做?

\vspace{0.5em}
\noindent \textbf{6)科技强国与人文素养:}
你认为一个优秀的工程师/科学家,为什么也需要人文素养?请结合 AI、核能、基因编辑等例子说明。

\vspace{0.5em}
\noindent \textbf{7)善意 + 规则:}
如果同学经常借作业但不自己思考,你会怎么处理,才能既保持善意,又不鼓励“搭便车”?

\vspace{0.5em}
\noindent \textbf{8)图像寓意:数据与隐私:}
想象一幅画:一座“玻璃城市”,楼里每个人的行为都被数字气泡标注,城市上空写着“Better service”。这幅画可能在批评什么?你会如何回应这种担忧?

\vspace{0.5em}
\noindent \textbf{9)传统节日再设计:}
如果由你来重新设计一个“现代青年版中秋节”活动,你会保留原来的哪些元素?增加哪些新形式?为什么?

\vspace{0.5em}
\noindent \textbf{10)翻译并理解(焦虑时代):}
翻译并理解:\\
\textit{Not every notification deserves your attention, not every urgency deserves your anxiety.}

\vspace{0.5em}
\noindent \textbf{11)美术作品与社会观察:}
一幅画中,城市霓虹下有一群低头看手机的人,只有远处一扇窗透出黄灯,一个人在读纸质书。你如何解读这幅画?它在提醒我们什么?

\vspace{0.5em}
\noindent \textbf{12)文化与算法推荐:}
越来越多的传统艺术(昆曲、评弹、戏剧)依靠短视频推荐“出圈”。你认为这会改变传统艺术的形态吗?利弊是什么?

\vspace{0.5em}
\noindent \textbf{13)大国博弈中的普通人:}
在“芯片战”“科技战”“贸易限制”的背景下,普通人的生活会受到哪些影响?作为学生,你现阶段能做的最现实的事情是什么?

\vspace{0.5em}
\noindent \textbf{14)世界冲突与同理心:}
看到战争/冲突新闻时,你更关注哪一类信息(军事、平民、外交、经济)?这反映了你怎样的价值观?

\vspace{1em}
\noindent \textbf{### 科学案例(7题)}

\noindent \textbf{1)极端天气与气候变化:}
近几年极端高温、暴雨频发。你认为这是“天气波动”还是“气候变化”的体现?如何区分?请列出至少两类科学证据。

\vspace{0.5em}
\noindent \textbf{2)轨道交通与能耗:}
地铁、高铁、私家车三种出行方式,你从物理和工程角度比较它们的能耗、效率和环境影响。

\vspace{0.5em}
\noindent \textbf{3)设计实验:塑料污染:}
设计一个简单实验,研究“塑料垃圾在阳光和雨水作用下是否会释放有害物质到水中”。说明控制变量与观测指标。

\vspace{0.5em}
\noindent \textbf{4)可再生能源组合:}
在一个海边城市,如果只能选择“海上风电 + 光伏 + 抽水蓄能”,你会怎样组合以保证稳定供电?请简要说明物理原理和优缺点。

\vspace{0.5em}
\noindent \textbf{5)城市热岛效应:}
为什么城市比周边农村更热?从热容量、反照率、能量输入三个方面解释。你认为有哪些工程/规划办法减缓这一问题?

\vspace{0.5em}
\noindent \textbf{6)疫苗与群体免疫:}
简单解释疫苗的免疫原理(不要求专业名词),以及“群体免疫”大致是什么意思。你如何看待“个人选择”与“公共健康”的平衡?

\vspace{0.5em}
\noindent \textbf{7)微塑料与生命系统:}
科学家在海洋、雪地甚至人体组织中发现微塑料。你认为它们可能通过什么路径进入人体?从“规模、累积效应、不确定性”三个角度谈谈你对这一问题的判断。

\vspace{1em}
\noindent \textbf{### 小组辩论(8题)}

\noindent \textbf{1)成绩公开:}班级是否应该公开每次考试的详细成绩排名?\\
\textbf{2)手机号实名制:}手机和网络账号必须实名,对社会是利大于弊还是弊大于利?\\
\textbf{3)“内卷”与“躺平”:}面对学习和竞争,“努力拼搏”与“适当松弛”,哪一种更适合作为长期生活态度?\\
\textbf{4)素质教育与应试教育:}在当前环境下,应试训练和素质拓展,哪个更应优先?\\
\textbf{5)短视频平台监管:}对于未成年人使用短视频平台,应该更强调“家长责任”还是“平台责任”?\\
\textbf{6)校园 AI 使用规范:}学校是否应该禁止在作业中使用生成式 AI?\\
\textbf{7)名校情结:}追求“名校”是动力还是枷锁?对个体成长利大于弊还是弊大于利?\\
\textbf{8)消费降级/升级:}在经济压力下,“理性消费”(降级)与“为喜欢的东西买单”(升级),哪一种更能代表健康的生活态度?

\vspace{1em}
\noindent \textbf{### 小心理测试(7题)}

\noindent \textbf{1)面对否定:}当一个你尊敬的人否定了你的想法,你第一反应通常是什么?你如何处理这种情绪?\\
\textbf{2)拖延原因:}说一件你总想做却总拖延的事。你觉得真正的原因是什么?(怕失败、没有计划、奖励不足、环境干扰……)\\
\textbf{3)情绪调节:}当你连续几天心情都不好时,你通常会做什么来调整自己?\\
\textbf{4)社交角色:}在新环境中(例如新班级),你更倾向于主动认识别人还是等待别人来认识你?为什么?\\
\textbf{5)被误解的经历:}讲一次你被误解的经历,当时你是如何澄清的?如果再来一次,你会有什么不一样的做法?\\
\textbf{6)完美主义 vs 完成主义:}做作业或项目时,你更在意“做到完美”还是“按时完成”?这给你带来了哪些好处和困扰?\\
\textbf{7)未来自我:}如果五年后你可以给现在的自己发一条消息,你希望那条消息写什么?

\vspace{1em}
\noindent \textbf{### 小组讨论(7题)}

\noindent \textbf{1)兴趣与功利:}当“兴趣”与“对升学最有利的选择”冲突时,你如何权衡?请结合你已经做过或即将做的一个选择来谈。\\
\textbf{2)失败叙事:}每个人轮流讲一个失败经历,之后小组一起讨论:这些失败中有哪些共通原因?如果组成一个“反失败手册”,第一条会写什么?\\
\textbf{3)校园规则共创:}如果让你们小组为学校提出一条“新的校园规则”,你们会从哪个问题出发?规则大致内容是什么?\\
\textbf{4)数字生活习惯:}分享你手机上使用时间最长的三个 App,讨论它们分别带给你什么价值和风险。\\
\textbf{5)理想课堂:}如果你可以重新设计一节物理/数学课的上课方式,你会怎么设计?为谁服务?如何评价效果?\\
\textbf{6)危机应对:}假设学校临时停课一周,你们小组要为“如何保证这一周学习效率”做一个方案。你会从哪些方面入手?\\
\textbf{7)团队分歧:}当小组内部对一个方案意见不一致时,你认为最好的决策流程是什么样的?

\vspace{1em}
\textbf{4. 必背物理常识(结合热点):}
\begin{itemize}
    \item \textbf{能量守恒:}所有永动机(包括某些号称无需充电的AI设备)都是骗局。
    \item \textbf{波粒二象性:}量子现象与信息技术的基础背景。
    \item \textbf{熵增原理:}系统趋于无序;信息处理与能耗、散热问题常可联系“熵”的概念进行解释。
\end{itemize}

\section*{西交大少年班面试特别指南3}

\textbf{1. 遇到不懂的问题怎么办?}

\noindent 不要乱编。可以这样回答:“这个具体的知识点我目前没有涉猎,但我可以根据我学过的知识尝试分析一下...” —— \textbf{展示思维过程比展示结果更重要。}

\vspace{1em}
\textbf{2. 如何体现“少年班”特质?}

\noindent 少年班寻找的是“早慧”且“潜力大”的学生。回答时要做到:\textbf{观点清晰}、\textbf{逻辑完整}、\textbf{敢于提出假设并验证}。遇到开放题可用“现象—原因—影响—对策”的结构组织语言。

\vspace{1em}
\textbf{3. 面试题库(版本三:文化自信与中国实践,2025-2026全球热点):}

\vspace{0.5em}
\noindent \textbf{### 人文案例(14题)}

\noindent \textbf{1)诗句与自我期许:}
“俱怀逸兴壮思飞,欲上青天览明月。”谈你理解中的“壮思飞”,并联系你心中最想尝试的一件“看起来有点冒险”的事情。

\vspace{0.5em}
\noindent \textbf{2)家国情怀:}
“先天下之忧而忧,后天下之乐而乐。”在今天如何理解“忧”和“乐”?你觉得中学生如何在日常生活中体现这种情怀?

\vspace{0.5em}
\noindent \textbf{3)城乡差异与教育公平:}
如果你有机会去乡村支教一周,你最想带给当地学生的是什么?你觉得“优质教育资源不均”带来的最大问题是什么?

\vspace{0.5em}
\noindent \textbf{4)乡村振兴与文化保护:}
许多古村落通过民宿、文创、短视频被“带火”。你如何看待“保护原貌”和“发展旅游”的矛盾?你会如何平衡?

\vspace{0.5em}
\noindent \textbf{5)大运会/亚运会/奥运会观感:}
大型体育赛事中,你更关注“金牌数量”还是“运动员故事”?这反映了你怎样的价值选择?

\vspace{0.5em}
\noindent \textbf{6)城市更新与记忆:}
老城区改造时,有人主张“全部拆掉重建现代化”,有人主张“最大限度保留老建筑”。你赞成哪一方?为什么?

\vspace{0.5em}
\noindent \textbf{7)网络热点与理性:}
当某个社会事件在网络上引发巨大情绪时,你认为“第一时间站队”重要,还是“冷静等待更多信息”更重要?请举一个你关注过的事件说明。

\vspace{0.5em}
\noindent \textbf{8)图像寓意:中国制造:}
想象一幅画:一条生产线上,一端是复杂芯片和高端设备,另一端是普通生活用品,墙上写着“Made in China 2.0”。你如何理解这幅画传递的含义?

\vspace{0.5em}
\noindent \textbf{9)文化自信与批判精神:}
你如何理解“文化自信”与“盲目自大”的区别?当你看到自己国家的问题时,如何做到既不自卑也不自满?

\vspace{0.5em}
\noindent \textbf{10)翻译并理解(走出舒适区):}
翻译并理解:\\
\textit{Growth begins at the edge of your comfort zone, but it doesn't mean you have to jump off a cliff.}

\vspace{0.5em}
\noindent \textbf{11)影视作品与价值观:}
选择一部你最近看过的国产电影/电视剧/纪录片,谈谈其中一个让你印象最深的角色或情节,以及它对你的启发。

\vspace{0.5em}
\noindent \textbf{12)传统仪式与现代生活:}
清明、七夕、重阳等节日中,你觉得哪一个最有现代延展空间?你会如何用现代方式重新表达其中的情感?

\vspace{0.5em}
\noindent \textbf{13)全球视野:他者之眼中的中国:}
你看到过哪些外国人拍的中国 vlog/纪录片?他们眼中的中国与你自己生活的中国有什么相同或不同?

\vspace{0.5em}
\noindent \textbf{14)科技民族主义与开放合作:}
当看到“中国科技突破/某领域反超”的新闻时,你有什么感受?你如何看待“在竞争中合作、在合作中竞争”的国际关系?

\vspace{1em}
\noindent \textbf{### 科学案例(7题)}

\noindent \textbf{1)大工程与集体智慧:}
高铁、港珠澳大桥、南水北调等大工程背后,你认为最关键的科学/工程挑战各是什么?选择一个例子简要说明。

\vspace{0.5em}
\noindent \textbf{2)可再生能源在中国:}
风能、太阳能、水电在中国的发展有什么区域特点?请选择其中一个,从物理原理和地理条件两方面说明。

\vspace{0.5em}
\noindent \textbf{3)卫星导航系统:}
简要说明卫星导航(如北斗)的基本原理:至少提到“卫星位置”“时间同步”“三维定位”。你觉得这一系统对普通人生活改变最大的地方在哪里?

\vspace{0.5em}
\noindent \textbf{4)空气质量与健康:}
PM2.5 是什么?它为什么危险?如果让你设计一个校园“小实验/小调查”来关注空气质量,你会怎么做?

\vspace{0.5em}
\noindent \textbf{5)粮食安全与科技:}
从科学角度看,保证粮食安全有哪些关键技术(例如品种改良、灌溉、土壤改良、储存运输等)?你觉得哪一方面风险最大?

\vspace{0.5em}
\noindent \textbf{6)公共卫生与行为科学:}
在传染病防控中,除了药物和设备,“人们的行为”也很重要。举一个你认为最重要的行为改变,并从科学或心理角度解释其作用。

\vspace{0.5em}
\noindent \textbf{7)海洋科技与未来:}
深海采矿、海洋风电、海洋牧场、深海探测器……你选一个方向,说明它的科学/工程难点以及可能带来的机遇与风险。

\vspace{1em}
\noindent \textbf{### 小组辩论(8题)}

\noindent \textbf{1)城市发展重心:}一个城市应该优先发展“高科技产业”还是优先改善“宜居环境”?\\
\textbf{2)乡村直播带货:}农产品/乡村通过直播带货,对乡村振兴是利大于弊还是弊大于利?\\
\textbf{3)中学生是否需要“职业规划”?}\\
\textbf{4)家长群行为:}家长在班级群中过多干预班级事务,是利大于弊还是弊大于利?\\
\textbf{5)校园竞赛热:}各种学科竞赛、活动是否会加剧“内卷”?是否应该适当限制?\\
\textbf{6)校服统一:}规定统一校服,对学生发展是利大于弊还是弊大于利?\\
\textbf{7)“读万卷书”与“行万里路”:}在中学阶段哪一个更重要?\\
\textbf{8)网络民族主义:}在互联网上表达爱国情绪的方式,是否需要更审慎?

\vspace{1em}
\noindent \textbf{### 小心理测试(7题)}

\noindent \textbf{1)改变的难度:}如果让你在“每天早起半小时”和“每天多运动半小时”之间选一个坚持一个月,你会选哪一个?为什么?\\
\textbf{2)被夸奖的方式:}别人怎样的夸奖/认可最能让你感觉被“真正看见”?举一个场景。\\
\textbf{3)冲突处理风格:}当你和好朋友发生严重分歧时,你通常会怎样做:冷处理、直接对话、找第三方、写下来?为什么?\\
\textbf{4)规则与自由:}你最反感的一条校园规则是什么?它背后的合理性和问题分别是什么?\\
\textbf{5)深度专注:}你最近一次“完全沉浸在一件事里忘记时间”是什么时候?在做什么?\\
\textbf{6)自我安慰方式:}当你遇到挫折时,你心里会对自己说什么话?这句话对你有用吗?\\
\textbf{7)群体归属感:}在家庭、班级、兴趣社团、网络社群中,你对哪一个群体的归属感最强?为什么?

\vspace{1em}
\noindent \textbf{### 小组讨论(7题)}

\noindent \textbf{1)班级氛围:}你们理想中的班级氛围是什么样的?请从“学习、关系、规则”三个角度描绘。\\
\textbf{2)校园公共空间设计:}如果给你们一块空地做“学生自主公共空间”,你们会怎样规划?(安静区/讨论区/运动区/展览区……)\\
\textbf{3)学习方式的差异:}小组成员分享各自最有效的一种学习习惯,讨论这些习惯中有哪些共性?能否形成一份“经验清单”?\\
\textbf{4)手机进校园:}你们小组内部先达成一个“手机在校园使用的准则”,再向考官简要陈述你们的共识与分歧点。\\
\textbf{5)错误的价值:}分享一次“通过犯错学到东西”的经历,讨论:学校应该如何对待学生的错误?\\
\textbf{6)未来城市想象:}共同构想一座“未来城市”(20年后),从交通、能源、教育、环境四个方面给出你们的创意设想。\\
\textbf{7)跨年级合作:}如果你可以设计一个“初高中跨年级项目”,你会选择什么主题?大致怎样分工?

\vspace{1em}
\textbf{4. 必背物理常识(结合热点):}
\begin{itemize}
    \item \textbf{能量守恒:}所有永动机(包括某些号称无需充电的AI设备)都是骗局。
    \item \textbf{波粒二象性:}量子现象与信息技术的基础背景。
    \item \textbf{熵增原理:}系统趋于无序;信息处理与能耗、散热问题常可联系“熵”的概念进行解释。
\end{itemize}

\section*{西交大少年班面试特别指南4}

\textbf{1. 遇到不懂的问题怎么办?}

\noindent 不要乱编。可以这样回答:“这个具体的知识点我目前没有涉猎,但我可以根据我学过的知识尝试分析一下...” —— \textbf{展示思维过程比展示结果更重要。}

\vspace{1em}
\textbf{2. 如何体现“少年班”特质?}

\noindent 少年班寻找的是“早慧”且“潜力大”的学生。回答时要做到:\textbf{观点清晰}、\textbf{逻辑完整}、\textbf{敢于提出假设并验证}。遇到开放题可用“现象—原因—影响—对策”的结构组织语言。

\vspace{1em}
\textbf{3. 面试题库(版本四:个人成长与数智社会,2025-2026全球热点):}

\vspace{0.5em}
\noindent \textbf{### 人文案例(14题)}

\noindent \textbf{1)诗句与自控力:}
“欲穷千里目,更上一层楼。”联系一次你“强迫自己多走一步”的经历,这一步给你带来了什么变化?

\vspace{0.5em}
\noindent \textbf{2)时间价值:}
“少壮不努力,老大徒伤悲。”在今天是否还适用?你如何看待“早一点努力”和“晚一点觉醒”?

\vspace{0.5em}
\noindent \textbf{3)代际理解:}
如果你的父母不理解你在网上花时间学习/创作,你会如何解释?这背后反映了怎样的代际差异?

\vspace{0.5em}
\noindent \textbf{4)网络人设与真实自我:}
你在网络上的形象和现实中的自己有哪些不同?这些不同更接近“理想中的自己”还是“迎合他人的自己”?

\vspace{0.5em}
\noindent \textbf{5)校园舆论与孤立:}
一次校园冲突后,舆论开始一边倒地指责某位同学。你如何避免被“情绪洪流”带着走?你会如何对待这位同学?

\vspace{0.5em}
\noindent \textbf{6)数据驱动社会:}
现在很多决策依赖“数据指标”(点击量、点赞数、转发量)。你觉得这一趋势有哪些好处和隐患?

\vspace{0.5em}
\noindent \textbf{7)自媒体时代的“发声权”:}
每个人都有麦克风,是好事还是坏事?你会如何使用自己的发声权?

\vspace{0.5em}
\noindent \textbf{8)图像寓意:被评分的人生:}
一幅画:人头上漂浮着一个评分条,从 0 到 100,不同场合不断变化。你如何理解这幅画?它在批评什么?你赞同吗?

\vspace{0.5em}
\noindent \textbf{9)数字鸿沟:}
当一些人可以轻松使用AI、云资源、优质网课,而另一些人连稳定网络都难以保障时,你认为这会带来什么新的不平等?

\vspace{0.5em}
\noindent \textbf{10)翻译并理解(算法时代):}
翻译并理解:\\
\textit{Algorithms can rank your choices, but they should never replace your values.}

\vspace{0.5em}
\noindent \textbf{11)职业想象:}
除了“工程师/科学家/医生/老师”等传统职业,你能想象 10 年后会出现哪些新职业?你对哪一种最感兴趣?

\vspace{0.5em}
\noindent \textbf{12)失败简历:}
如果你要写一份“失败简历”(只写试过但没成功的事情),你觉得会有哪些条目?你会从这份简历里读到什么?

\vspace{0.5em}
\noindent \textbf{13)数字记忆与遗忘权:}
互联网上的信息很难被彻底删除。你如何看待“被永久记住”和“被遗忘的权利”之间的矛盾?

\vspace{0.5em}
\noindent \textbf{14)全球互联与地方身份:}
你在网上可以和世界各地的人连接,但生活圈又很本地化。你更认同自己是“XX城市的学生”还是“全球公民”?为什么?

\vspace{1em}
\noindent \textbf{### 科学案例(7题)}

\noindent \textbf{1)推荐算法与“相似性”:}
推荐系统是如何“猜到”你可能喜欢什么内容的?请用“特征—相似度—反馈”三个词解释基本原理(不要求公式)。

\vspace{0.5em}
\noindent \textbf{2)深度伪造与可信度:}
随着 AI 能生成以假乱真的图片和视频,我们怎样才能验证一段视频的真伪?你能想到哪些技术或制度办法?

\vspace{0.5em}
\noindent \textbf{3)碳足迹与个人选择:}
在日常生活中,你觉得哪三类行为最影响个人碳排放?如果你只能改变其中一类,你会优先改变哪一个?

\vspace{0.5em}
\noindent \textbf{4)社交媒体与注意力:}
用“刺激—反应—奖励”的框架解释为什么短视频容易让人停不下来。这和大脑的哪种机制有关?

\vspace{0.5em}
\noindent \textbf{5)大规模在线教育:}
从技术角度看,线上教育解决了什么问题?从学习效果角度看,又带来了什么新问题?

\vspace{0.5em}
\noindent \textbf{6)虚拟现实与真实感:}
VR/AR 如何模拟“真实感”?从视觉、听觉、运动感受三个方面谈谈关键技术点。

\vspace{0.5em}
\noindent \textbf{7)数字孪生城市:}
什么是“数字孪生城市”?你认为在交通管理、灾害预警、城市规划中,它分别能起什么作用?

\vspace{1em}
\noindent \textbf{### 小组辩论(8题)}

\noindent \textbf{1)成绩 vs 能力:}中学阶段应该更重视“分数”还是“综合能力”?\\
\textbf{2)线下社交 vs 线上社交:}对青少年的成长,哪一种更重要?\\
\textbf{3)匿名制度:}网络评论是否应该全面取消匿名?\\
\textbf{4)AI 作文/绘画工具:}中学生是否应该自由使用这类工具完成作品?\\
\textbf{5)作业量:}适量作业是必要训练还是无效负担?\\
\textbf{6)课外补习:}是否应该严格限制校外培训机构?\\
\textbf{7)校园监控摄像头:}为了安全在校园大量安装摄像头是否合适?\\
\textbf{8)“早恋”管理:}学校是否应该制定严格的“禁止早恋”规定?

\vspace{1em}
\noindent \textbf{### 小心理测试(7题)}

\noindent \textbf{1)最在意谁的评价:}家人、同学、老师、网友中,你最在意哪一类人的评价?为什么?\\
\textbf{2)“被看见”的需求:}什么时候你会觉得自己“完全不被理解”?你一般会怎么处理这种感受?\\
\textbf{3)时间感:}对你来说,“一天很长,一年很短”还是“一天很短,一年很长”?为什么?\\
\textbf{4)冲突回避:}你有因为害怕冲突而不表达真实想法的时候吗?举一个例子。\\
\textbf{5)自我奖励机制:}完成一件困难的任务后,你会如何奖励自己?\\
\textbf{6)安全感来源:}当你感到不安时,什么事/谁最能给你安全感?\\
\textbf{7)改变环境 or 改变自己:}遇到不喜欢的环境,你更倾向于改变环境还是适应环境?举例说明。

\vspace{1em}
\noindent \textbf{### 小组讨论(7题)}

\noindent \textbf{1)数字断舍离:}如果让你们小组共同设计一个“7天数字减负挑战”(减少无意义刷屏),你们会制定哪些规则?\\
\textbf{2)学习互助小组:}如何设计一个真正有效的“互助小组”,避免变成“抄作业小组”?\\
\textbf{3)校园空间中的“安静角落”:}学校是否需要专门设计“安静学习/放空空间”?你们会怎么设计?\\
\textbf{4)同学间的差异:}你们小组讨论:你们彼此最不一样的一点是什么?这种差异在合作中是障碍还是资源?\\
\textbf{5)对未来的不确定感:}每个人描述自己最担心的一个未来问题,小组讨论这些担心有没有共性?能否提出一条“共同应对策略”?\\
\textbf{6)学生参与学校管理:}如果学校设立“学生议事会”,你最想推动的一项改变是什么?如何实现?\\
\textbf{7)理想的一天:}小组共同设计一份“理想中但现实可行”的中学生作息时间表,并说明设计理由。

\vspace{1em}
\textbf{4. 必背物理常识(结合热点):}
\begin{itemize}
    \item \textbf{能量守恒:}所有永动机(包括某些号称无需充电的AI设备)都是骗局。
    \item \textbf{波粒二象性:}量子现象与信息技术的基础背景。
    \item \textbf{熵增原理:}系统趋于无序;信息处理与能耗、散热问题常可联系“熵”的概念进行解释。
\end{itemize}